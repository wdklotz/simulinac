\documentclass[10pt]{article}
\usepackage{amsmath,amsfonts,amssymb,graphicx,parskip} 
\title{SIMULINAC User's Guide\\ v7.0.1}
\author{W.D. Klotz, wdklotz@alceli.com}
\begin{document}
\maketitle

\section{Getting Started}
SIMULINAC is a set of pure Python3 modules to simulate proton dynamics in a LINAC lattice.
It has two main modules \emph{simu.py} and \emph{tracker.py}. \emph{simu.py} is an envelope code and \emph{traker.py} is a tracking code. 

All files should be installed in a root directory like
\verb+$HOME/SIMULINAC+. The Python3 executable should be on your PATH environment variable. The Python installation must have \verb+matplotlib+, \verb+numpy+ and \verb+pyaml+ installed.

To start \emph{simu.py} type \fbox{python simu.py}. \\ To start \emph{tracker.py} type
\fbox{python tracker.py}.

Both will print on your terminal and before finishing display results in several 
figures.

\section{Preparing Input}
Input to the two programs has to be provided by a text file written in \verb+YAML+ syntax. 
The \emph{input}-file
is generated from a \emph{template}-file also written in \verb+YAML+. The \emph{template}-file contains immutable information and mutable information coded as \verb+m4+ macros. Before each 
program reads its \emph{input}-file it invokes the \verb+m4+ macro processor which reads a \emph{macro-definition}-file and replaces the macros in the \emph{template}-file.

\begin{center}
\begin{tabular}{c|c|c}
\hline
\multicolumn{3}{|c|} {standard file names} \\ \hline
\hline & simu.py & tracker.py \\
\hline macro-definitions & yml/macros.sh & yml/macos.sh \\
\hline template & ymlworktmpl.yml & yml/worktmpl.yml \\
\hline input & yml/simuIN.yml & yml/trackerIN.yml \\
\hline
\end{tabular}
\end{center}

The standard files in the distribution are good examples to see how this works.

\subsection{Special Input Parameters}
In the current version some input parameters have to set directly in the Python code. The corresponding code lines are behind the \verb+if __name__ == '__main__':+ line. For
\emph{simu.py} these are:
\begin{verbatim}
if __name__ == '__main__':

    # launch m4 to fill macros in template file
    template_file = 'yml/worktmpl.yml'      # template file
    input_file    = 'yml/simuIN.yml'        # input file
    macros_file   = 'yml/macros.sh'         # macro definitions
\end{verbatim}
they allow to customize the standard file names.

In \emph{tracker.py} the code is:
\begin{verbatim}
if __name__ == '__main__':
    :
    :    
    # launch m4 to fill macros in template file
    template_file = 'yml/worktmpl.yml'       # template file
    input_file    = 'yml/trackIN.yml'        # input file
    macros_file   = 'yml/macros.sh'          # macro definitions
    :    
    :    
    options = dict( input_file = input_file,
                    particles_per_bunch = 10000,
                    show    = True,
                    save    = False,
                    skip    = 1
\end{verbatim}
The standard file names, the \verb+particles_per_bunch+ and the two flags \verb+show+ and
\verb+save+ can be customized here. The \verb+skip+-flag is not used in this version.
The \verb+show+-flag is used to switch fugures on/off. The \verb+save+-flag is supposed
to save figures but is not working in this version.

\section{Input File Structure}
In this section the structure of the standard \emph{input}-file is discussed.
\subsection{Flags}
The \emph{flags}-block is used to select different features the programs provide. Both 
programs have access to the flag-settings but make different use of them or ignore them.
The current set of flags is given below. The \verb+{value}+s are the default values. To change from default values the corresponding line in the input-file has to be uncommented.

\begin{verbatim}
flags:
    # - accON:        False           # {True} acceleration on/off flag
    # - egf:          True            # {False} emittance growth flag
    # - sigma:        False           # {True} beam sizes by sigma-matrix
    # - KVout:        True            # {False} print a dictionary of Key-Value pairs
    # - periodic:     True            # {False} treat lattice as ring
    # - express:      True            # {False} use express version of thin quads
    # - useaper:      True            # {False} use aperture check for lattice elements
    # - csTrak:       False           # {True} plot CS trajectories
    # - bucket:       True            # {False} plot the bucket
    # - pspace:       True            # {False} plot the twiss ellipses at entrance
    # - verbose:      2               # {0} print flag (0 = minimal print), try 0,1,2,3
\end{verbatim}

Their use in both programs is tabeled below.
\begin{center}
\begin{tabular}{c|c|c}
flag & simu.py & tracker.py \\ \hline
\hline
accON & acceleration on/off & acceleration on/off \\
\hline
egf & emittance growth flag & emittance growth flag \\
\hline
sigma & beam sizes by sigma-matrix & not used\\
\hline
KVout & no display, Key/Value dictionary only& not used \\
\hline
periodic & treat lattice as ring & meaningless \\
\hline
express & express version of thin quads & express version of thin quads \\
\hline
useaper & aperture check for lattice elements & not used\\
\hline
csTrack & plot CS trajectories & not used \\
\hline
bucket & plot the RF bucket & plot the RF bucket  \\
\hline
pspace & plot entrance twiss ellipses & not used \\
\hline
verbose & verbose level 0=minimal& not used \\
\hline
\end{tabular}
\end{center}
Notes:
\begin{itemize}
\item egf stands for emittance growth formula. For details see the TRACE 3-D Documentation\footnote{TRACE 3-D Documentation by K.R.Crandal, D.P.Rusthoi, 1997, Appendix F}.
\item sigma selects the way to calculate beam envelope sigmas. If True the sigma-matrix
method is used to calulate beam sizes\footnote{TRACE 3-D Documentation by K.R.Crandal, D.P.Rusthoi, 1997, Appendix A} else the standard formulas from twiss\footnote{see Snyder \& Courant therory} functions are used.
\item KVout = True prints a long dictionary of internal parameters and supresses garaphics output.
\item periodic is used for testing purposes to see if the linear matrices produce correct results.
\item express replaces thick quadrupoles by thin quadrupoles with up- and downstream 
drift space.
\item useapaer: in envelope calculation the transverse 1-sigma beam envelope is checked against aperture limitations.
\item pspace plots the two transverse sphase space ellipses to check for transverse beam matching.
\end{itemize}

\subsection{Sections}
The \emph{sections}-block can be used to divide the lattice into different sections. Sections are not mandatory. If commented the whole lattice is one single unnamed section. If sections are defined, each element\footnote{see Elements below} must be assigned a section.
\begin{verbatim}
sections:
- [&LE 5/30,  &HE 30/200]
\end{verbatim}
Note:
\begin{itemize}
\item {[}\&LE 5/30, \&HE 30/200{]} defines a list of section tags. \&LE defines a link that can
be referenced elsewhere in the \verb+YAML+-file as \textasteriskcentered LE. The entry behind `5/30' can be any name and defines the
 name of the section. In this example a shortcut for the section from 5 to 30 Mev.
\end{itemize}

\subsection{parameters}
The \emph{parameters}-block defines global parameters as key-value pairs. Some prameters are defined with 
fixed values others by macro-variables.
\begin{verbatim}
parameters:
    - Tkin:                 _TKIN       # [MeV] energy @ entrance (injection)
    - emitx_i:      &emx    _EMITX      # [m*rad] {x,x'} emittance @ entrance
    - emity_i:      &emy    _EMITY      # [m*rad] {y,y'} emittance @ entrance
    - emitw_i:      &emw    _EMITW      # [rad] {Dphi,w} emittance @ entrance
    - betax_i:      &btx    _BETAX      # [m] twiss beta @ entrance x
    - betay_i:      &bty    _BETAY      # [m] twiss beta @ entrance y
    - phi_sync:     &phs    _PHISY      # [deg] synchronous phase
    - alfax_i:              0.          # [1] twiss alpha x @ entrance
    - alfay_i:              0.          # [1] twiss alpha y @ entrance
    - frequency:    &p01    816.e+6     # [Hz] frequency
    - ql0:          &p02    0.10        # [m] quad-length
    - ql:           &p03    0.05        # [m] 1/2 quad-length
    - quad_bore:    &p04    0.011       # [m] quad bore radius
    - windings:             30          # [1] quad-coil windings
    - gap:          &p15    0.048       # [m] RF gap
    - n_sigma:              10          # [m] sigma aperture
    - aperture:             15.e-3      # [m] global aperture setting (default = None)
\end{verbatim}
Notes:
\begin{itemize}
\item Tkin: is the key for the kinetic energy at injection in [Mev]. Its value is the macro-name \_TKIN.
\item emitx\_i: is the transvese emmittance in x-plane in [$m*rad$]. Its value is the macro-name \_EMITX and
\&emx its link-id.
\item aperture has a fixed value of $15.*10^{-3}$
\end{itemize}

\subsection{elements}
The \emph{elements}-block defines the elements (a.k.a nodes) in the lattice.
\begin{verbatim}
elements:
# HE
    - D3:        &D3             # ID:&link
        - type:     D            # type:class
        - length:   0.08         # [m]
        - sec:      *HE          # section
    - D5:    &D5
        - type:     D
        - length:   0.022
        - sec:      *HE
    - QFH:   &QFH                # ID:&link
        - type:     QF           # type:class
        - length:   *p03         # [m]
        - aperture: *p04         # [m] quad bore
        - B':       &Bgrad   30. # [T/m] gradient
        - slices:   0            # slices
        - sec:      *HE          # section
    - QDH:   &QDH
        - type:     QD
        - length:   *p02
        - aperture: *p04
        - B':       *Bgrad
        - slices:   0
        - sec:      *HE
    - RFGH:  &RFGH               # ID:&link
        - type:     RFG          # type:class
        - EzAvg:    1.00         # [MV/m] average E-field
        - EzPeak:   1.40         # [MV/m] peak E-field
        - PhiSync:  *phs         # [deg] synchronous phase
        - fRF:      *p01         # [Hz] frequency
        - gap:      *p15         # [m] length
        - aperture: *p04         # [m] quad bore
        - aperture: 10.e-3       # [m] quad bore
        - SFdata:   SF_WDK2g44.TBL # superfish tbl-data file
        - mapping:   t3d         # Trace 3D linear map model
        - mapping:   simple      # Shishlo/Holmes linear map model
        - mapping:   base        # Shishlo/Holmes base map model
        # - mapping:   ttf       # Shishlo/Holmes three point TTF RF gap-model
        # - mapping:   dyn       # Tanke/Valero RF gap-model
        :
        :
\end{verbatim}
Notes:
\begin{itemize}
\item
\begin{verbatim}
    - D3:        &D3             # the node-ID is D3, link-id is &D3
        - type:     D            # dipole node
        - length:   0.08         # [m]
        - sec:      *HE          # section
\end{verbatim}
type:D defines the node as an object of class `D', which is a dipole. length:0.08 defines its length in [m] and sec:\textasteriskcentered HE
says that this dipole belongs to section `HE' by reference.
\item
\begin{verbatim}
    - QFH:   &QFH                # ID:&link
        - type:     QF           # type:class
        - length:   *p03         # [m]
        - aperture: *p04         # [m] quad bore
        - B':       &Bgrad   30. # [T/m] gradient
        - slices:   0            # slices
        - sec:      *HE          # section
\end{verbatim}
The type:QF defines an x-focussing quadrupole. It has attributes length, aperture, B', slices and belongs to a section. 
\textasteriskcentered p03 and \textasteriskcentered p04 are references to links. B' has the value of 30. [$T/m$] and defines a link-id \&Bgrad. The slices attribute defines the number of slices a thick node is cut into. slices = 0 or 1 means don't slice the thick
node. slices = n means means cut it into n slices. The sec attribute references the HE section.
\item
\begin{verbatim}
    - RFGH:  &RFGH               # ID:&link
        - type:     RFG          # type:class
        - EzAvg:    1.00         # [MV/m] average E-field on axis
        - EzPeak:   1.40         # [MV/m] (EzAvg = 1.00[MV/m])
        - PhiSync:  *phs         # [deg] synchronous phase
        - fRF:      *p01         # [Hz] frequency
        - gap:      *p15         # [m] length
        - aperture: 10.e-3       # [m]
        - SFdata:   SF_WDK2g44.TBL # superfish tbl-data file
        - mapping:   t3d         # Trace 3D linear map model
        - mapping:   simple      # Shishlo/Holmes linear map model
        - mapping:   base        # Shishlo/Holmes base map model
        # - mapping:   ttf       # Shishlo/Holmes three point TTF RF gap-model
        # - mapping:   dyn       # Tanke/Valero RF gap-model
        - sec:      *HE          # section
\end{verbatim}
The type:RFC defines a radio frequency cavity. It is modeled as a kick of zero length. EzAvg is the average
E-field on the axis E(r=0,z). PhiSync is the synchronous phase in [deg]. It takes its value from the link
reference \textasteriskcentered phs. fRF is the rf frequency in [Hz] given here from the link reference 
\textasteriskcentered p15. aperture is the cavity bore radius. SFdata is the file name of a table for
field profile data from `SuperFish'. The mapping attribute specifies which cavity model to use. The cavity
node is part of section HE - sec attribute.

\subsection{segments}
\begin{verbatim}
segments:
# LE            # empty section
# HE            # high energy section
    - SEG1H:
        - *QFH
        - *D3
    - SEG2H:
        - *D3
        - *QDH
        - *D3
    - SEG3H:
        - *D3
        - *QFH
    - RFGH:
        - *D5
        - *RFGH
        - *D5
        #
        - *D5
        - *RFGH
        - *D5
        #
        - *D5
        - *RFGH
        - *D5
        #
        - *D5
        - *RFGH
        - *D5
        #
        - *D5
        - *RFGH
        - *D5
        #
        - *D5
        - *RFGH
        - *D5
        #
        - *D5
        - *RFGH
        - *D5
        #
        - *D5
        - *RFGH
        - *D5
        #
        - *D5
        - *RFGH
        - *D5
        #
        # - *D5     # 10th cavity makes it unstable!!
        # - *RFGH
        # - *D5
        #
\end{verbatim}
\end{itemize}
\end{document}